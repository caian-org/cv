\newcommand{\cvfirstname}{Caian}
\newcommand{\cvlastname}{Ertl}

\newcommand{\cvlocation}{São Paulo, SP}
\newcommand{\cvemail}{hi@caian.org}

\newcommand{\cvgituser}{github.com/caian-org}
\newcommand{\cvgitlink}{https://{\cvgituser}}

\newcommand{\cvskillsa}{Python, JS/TS (Node), Golang, C, Shell, {\LaTeX}}
\newcommand{\cvskillsb}{Ansible, Terraform, Docker, GitLab-CI, Azure DevOps}
\newcommand{\cvskillsc}{AWS, Azure, Heroku}
\newcommand{\cvskillsd}{MongoDB, PostgreSQL}


% title and contact
\newcommand{\cvtitle}{Desenvolvedor Backend}

\newcommand{\cvintrosecname}{Resumo}

\newcommand{\cvintro}{
    Experimente no desenvolvimento de aplicações web, pipelines de CI/CD,
    conteinerização, infraestrutura como código e arquiteturas em nuvem.
    Familiaridade com transformação digital, metodologias ágeis e cultura
    DevOps. Interessado em programação de alto desempenho, arquiteturas
    escaláveis e sistemas resilientes.
}
%----------------------------------------------------------------------------------------


% intro, skills, technologies
\newcommand{\cvskillseca}{Linguagens}
\newcommand{\cvskillsecb}{DevOps}
\newcommand{\cvskillsecc}{Nuvens}
\newcommand{\cvskillsecd}{Banco de Dados}
%----------------------------------------------------------------------------------------


% experience
\newcommand{\cvjobexpsecname}{Experiência}

% job mz
\newcommand{\cvjobperiodmz}{10/2019 -- presente}
\newcommand{\cvjobtitlemz}{Desenvolvedor Backend}
\newcommand{\cvjobcompanymz}{MZ Group}
\newcommand{\cvjobdescriptionmz}{
    A MZ é uma consultoria especializada em RI que oferece uma suite completa de
    soluções para empresas públicas.\newline

    Atuação dedicada ao MZiQ, a suite de produtividade com funcionalidades como
    gestão de relacionamento com investidores, gerenciamento de base acionária,
    dados públicos de instituições financeiras e fundos, geração de relatórios,
    insights etc. Atuo no desenvolvimento de novas funcionalidades, correção de
    bugs, automações diversas, criação de novas APIs e serviços e desenho de
    arquitetura e programação de sistemas baseados em mensageria e
    microsserviços. Também faço extração de dados em bancos Postgres e Mongo e
    criação de pipelines com Gitlab-CI. Extensivo uso de tecnologias como JS,
    TS, Python, RabbitMQ, AWS Lambda e Docker.
}

% job a
\newcommand{\cvjobperioda}{05/2019 -- 10/2019}
\newcommand{\cvjobtitlea}{Desenvolvedor de Chatbot}
\newcommand{\cvjobcompanya}{HVAR Consultoria}
\newcommand{\cvjobdescriptiona}{
    A HVAR oferece serviços e produtos orientados a big data, analytics e IA. É
    parceira SAP, Google Cloud e AWS no Brasil.\newline

    Participação no time de IA no desenvolvimento de chatbots usando a stack de
    tecnologias da Microsoft (Azure Bot Service, LUIS e CosmoDB), Google
    (Dialogflow e Firebase) e IBM (Watson). Atuei em projetos no entendimento
    das regras de negócio, construção de diálogos, fluxos de conversa e
    desenvolvimento de microsserviços e webservices que faziam uso dos serviços
    de cognição -- dos providers acima -- e APIs de terceiros a fim de
    automatizar tarefas, atendimento ao cliente etc. Também escrevi testes
    unitários e de integração para validação dos diálogos, pipelines de CI/CD
    com Azure DevOps e telemetria das conversas para geração de métricas de uso
    e taxa de acerto dos bots.
}

% job b
\newcommand{\cvjobperiodb}{02/2018 -- 05/2019}
\newcommand{\cvjobtitleb}{Desenvolvedor de Infraestrutura}
\newcommand{\cvjobcompanyb}{Rivendel Tecnologia}
\newcommand{\cvjobdescriptionb}{
    A Rivendel é especialista em computação em nuvem, DevOps, engenharia de
    dados e transformação digital.\newline

    No time de monitoramento, desenvolvi ferramentas e automações em Python
    para auxiliar no monitoramento de sistemas e aplicações, bem como um
    dashboard real-time em JS (com uso de sockets, jQuery, Redis e outras
    tecnologias) para agregação de métricas de monitoramento dos servidores
    Zabbix e serviços como PagerDuty e Pingdom.\newline

    No time de implantação, atuei em diversos projetos -- remoto e on-site --
    desenvolvendo microsserviços com AWS Lambda (em Python, JS, Go e Java),
    pipelines de CI/CD (Travis, Drone-CI, Jenkins etc) e provisionando
    infraestrutura em nuvem como código com Terraform e configuração como
    código com Ansible. Também desenvolvi POCs em linguagens como Ruby e C\# a
    fim de exemplificar, no contexto da stack de tecnologias do cliente, o uso
    de gitflow, pipelines, integração com serviços gerenciados da AWS e outros.
}
%----------------------------------------------------------------------------------------


% education
\newcommand{\cveducationsecname}{Formação}

% course mba
\newcommand{\cvcourseperiodusp}{10/2021 -- 04/2023}
\newcommand{\cvcoursetitleusp}{MBA em Data Science e Analytics}
\newcommand{\cvcourseinstitutionusp}{Universidade de São Paulo (USP)}
\newcommand{\cvcoursedescriptionusp}{
    Pós-graduação focada em big data, aprendizado de máquina, engenharia /
    mineração / visualização de dados.
}

% course a
\newcommand{\cvcourseperioda}{2015 -- 2018}
\newcommand{\cvcoursetitlea}{Bacharelado em Sistemas de Informação}
\newcommand{\cvcourseinstitutiona}{Universidade Anhembi Morumbi (AUM)}
\newcommand{\cvcoursedescriptiona}{
    Melhor avaliado em: construção de algorítmos, linguagens de programação,
    gestão de projetos e sistemas distribuídos.
}
