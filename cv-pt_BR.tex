%%%%%%%%%%%%%%%%%%%%%%%%%%%%%%%%%%%%%%%%%
%
% Developer CV
% LaTeX Template
% Version 1.0 (28/1/19)
%
% This template originates from:
% http://www.LaTeXTemplates.com
%
% Authors:
% Jan Vorisek (jan@vorisek.me)
% Based on a template by Jan Küster (info@jankuester.com)
% Modified for LaTeX Templates by Vel (vel@LaTeXTemplates.com)
%
% License:
% The MIT License (see included LICENSE file)
%
%%%%%%%%%%%%%%%%%%%%%%%%%%%%%%%%%%%%%%%%%

%----------------------------------------------------------------------------------------
%	PACKAGES AND OTHER DOCUMENT CONFIGURATIONS
%----------------------------------------------------------------------------------------

\documentclass[9pt]{developercv} % Default font size, values from 8-12pt are recommended

%----------------------------------------------------------------------------------------

\begin{document}


%----------------------------------------------------------------------------------------
%	TITLE AND CONTACT INFORMATION
%----------------------------------------------------------------------------------------

\newcommand{\gituser}{github.com/caian-org}
\newcommand{\gitlink}{https://{\gituser}}

\begin{minipage}[t]{0.725\textwidth} % 45% of the page width for name
	\vspace{-\baselineskip} % Required for vertically aligning minipages

	% If your name is very short, use just one of the lines below
	% If your name is very long, reduce the font size or make the minipage wider and reduce the others proportionately
	\colorbox{black}{{\HUGE\textcolor{white}{\textbf{\MakeUppercase{Caian}}}}} % First name

	\colorbox{black}{{\HUGE\textcolor{white}{\textbf{\MakeUppercase{Ertl}}}}} % Last name

	\vspace{6pt}

	{\huge Desenvolvedor Back-End} % Career or current job title
\end{minipage}
\begin{minipage}[t]{0.275\textwidth} % 27.5% of the page width for the first row of icons
	\vspace{-\baselineskip} % Required for vertically aligning minipages

	% The first parameter is the FontAwesome icon name, the second is the box size and the third is the text
	% Other icons can be found by referring to fontawesome.pdf (supplied with the template) and using the word after \fa in the command for the icon you want
	\icon{MapMarker}{12}{São Paulo, SP}\\
	\icon{Phone}{12}{[REDACTED]}\\
	\icon{At}{12}{\href{mailto:hi@caian.org}{hi@caian.org}}\\
	\icon{Github}{12}{\href{{\gitlink}}{{\gituser}}}\\
\end{minipage}

\vspace{0.5cm}


%----------------------------------------------------------------------------------------
%	INTRODUCTION, SKILLS AND TECHNOLOGIES
%----------------------------------------------------------------------------------------

\cvsect{Resumo}

\begin{minipage}[t]{0.40\textwidth}
	\vspace{-\baselineskip} % Required for vertically aligning minipages

    {\setstretch{1.0}\small{
        Profissional com experiência no desenvolvimento de aplicações web,
        pipelines de CI/CD, conteinerização, infraestrutura como código e
        arquiteturas em nuvem. Familiaridade em transformação digital,
        metodologias ágeis e cultura DevOps. Interessado em programação de alto
        desempenho, arquiteturas escaláveis e sistemas resilientes.
    }}
\end{minipage}
\hfill
\begin{minipage}[t]{0.65\textwidth}
	\vspace{-\baselineskip} % Required for vertically aligning minipages

    \begin{skillset}
        \skill
            {Programação}
            {Python, JavaScript, Golang, C, Java, Crystal, Shell, {\LaTeX}}
        \skill
            {DevOps}
            {Ansible, Terraform, Docker, Jenkins, GitLab-CI, Azure DevOps}
        \skill
            {Cloud}
            {AWS, Azure, Heroku}
        \skill
            {Banco de dados}
            {MySQL / MariaDB, PostgreSQL}
    \end{skillset}

\end{minipage}

\vspace{0.3cm}


%----------------------------------------------------------------------------------------
%	EXPERIENCE
%----------------------------------------------------------------------------------------

\cvsect{Experiência Profissional}
\begin{entrylist}
	\entry
		{05/2019 -- atual}
		{Chatbot Developer}
		{Hvar Consulting}
		{
            Desenvolvimento de chatbots usando a stack de tecnologias do
            Microsoft Bot Framework e serviços Azure. Uso da SDK em JS para
            criação de chatbots com integração em serviços de cognição (como
            LUIS), telemetria com Application Insights e pipelines de CI/CD com
            Azure DevOps. Escrita de testes unitários e de integração para
            validação dos diálogos, fluxos de conversa e integrações com APIs
            de terceiros.
        }
	\entry
		{02/18 -- 05/19}
		{Infrastructure Developer}
		{Rivendel Tecnologia}
		{
            No time de monitoramento, desenvolvi utilitários e automações em
            Python para auxiliar no monitoramento de sistemas e aplicações.
            Também desenvolvi um dashboard real-time em JS (com uso de sockets,
            jQuery, Redis e outras tecnologias) para agregação de métricas de
            monitoramento dos servidores Zabbix e serviços como PagerDuty e
            Pingdom.\newline\newline
            No time de implantação, atuei em diversos projetos -- remoto e
            in-loco -- no desenvolvimento de microsserviços com AWS Lambda (em
            Python, JS e Java), construção de pipelines de CI/CD (Travis,
            Drone-CI, Jenkins etc) e provisionamento de infraestrutura em nuvem
            como código com Terraform e configuração como código com Ansible.
            Também desenvolvi POCs em linguagens como Ruby e C\# a fim de
            exemplificar, no contexto da stack de tecnologias do cliente, o uso
            de gitflow, pipelines, integração com serviços gerenciados da AWS e
            outros.
        }
	\entry
		{11/16 -- 03/17}
		{Solutions Engineer}
		{Alest Consultoria}
		{
            Participação no time de projetos onde auxiliei no desenvolvimento
            de aplicações baseadas em JS com Google AppMaker e integrações com
            as ferramentas de produtividade do G Suite e APIs do Google.
            Criação de macros com Google Apps Script e pequenos webservices em
            Python.
        }
\end{entrylist}


%----------------------------------------------------------------------------------------
%	EDUCATION
%----------------------------------------------------------------------------------------

\cvsect{Formação Acadêmica}

\begin{entrylist}
	\entry
		{2015 -- 2018}
		{Bacharelado em Sistemas de Informação}
		{Universidade Anhembi Morumbi}
		{
            Disciplinas de destaque: construção de algorítmos, linguagem de
            programação, laboratório de programação, gestão de projetos, sistemas
            distribuídos.
        }
\end{entrylist}


%----------------------------------------------------------------------------------------
%	ADDITIONAL INFORMATION
%----------------------------------------------------------------------------------------

\begin{minipage}[t]{0.7\textwidth}
	\vspace{-\baselineskip} % Required for vertically aligning minipages

	\cvsect{Projetos Paralelos}
    \begin{listf}
        \itemf
            {dora}
            {{\gituser}}
            {
                Microsserviço desenvolvido em Python para consulta de registros
                DNS de domínios a partir de uma API RESTful.
            }
        \itemf
            {msgboi}
            {{\gituser}}
            {
                Microsserviço desenvolvido em JS para notificação de eventos do
                GitLab-CI em canais do Slack. Customizável e extremamente
                pequeno.
            }
        \itemf
            {ansible-stow}
            {{\gituser}}
            {
                Módulo para o Ansible capaz de interagir com pacotes do GNU
                Stow.
            }
    \end{listf}
\end{minipage}
\hfill
\begin{minipage}[t]{0.2\textwidth}
	\vspace{-\baselineskip} % Required for vertically aligning minipages

	\cvsect{Idiomas}

    \begin{itemize}[leftmargin=0.5cm, topsep=0cm]
        \item \textbf{Português} -- Nativo
        \item \textbf{Inglês} -- Proficiente
    \end{itemize}
\end{minipage}

%----------------------------------------------------------------------------------------

\end{document}
